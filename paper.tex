%%%%%%%%%%%%%%%%%%%%%%% file typeinst.tex %%%%%%%%%%%%%%%%%%%%%%%%%
%
% This is the LaTeX source for the instructions to authors using
% the LaTeX document class 'llncs.cls' for contributions to
% the Lecture Notes in Computer Sciences series.
% http://www.springer.com/lncs       Springer Heidelberg 2006/05/04
%
% It may be used as a template for your own input - copy it
% to a new file with a new name and use it as the basis
% for your article.
%
% NB: the document class 'llncs' has its own and detailed documentation, see
% ftp://ftp.springer.de/data/pubftp/pub/tex/latex/llncs/latex2e/llncsdoc.pdf
%
%%%%%%%%%%%%%%%%%%%%%%%%%%%%%%%%%%%%%%%%%%%%%%%%%%%%%%%%%%%%%%%%%%%


\documentclass[runningheads,a4paper]{llncs}

\usepackage{amssymb}
\setcounter{tocdepth}{3}
\usepackage{graphicx}

\usepackage{url}
\urldef{\mailsa}\path|{alfred.hofmann, ursula.barth, ingrid.haas, frank.holzwarth,|
\urldef{\mailsb}\path|anna.kramer, leonie.kunz, christine.reiss, nicole.sator,|
\urldef{\mailsc}\path|erika.siebert-cole, peter.strasser, lncs}@springer.com|    
\newcommand{\keywords}[1]{\par\addvspace\baselineskip
\noindent\keywordname\enspace\ignorespaces#1}

\begin{document}

\mainmatter  % start of an individual contribution

% first the title is needed
\title{Lecture Notes in Computer Science:\\Authors' Instructions
for the Preparation\\of Camera-Ready
Contributions\\to LNCS/LNAI/LNBI Proceedings}

% a short form should be given in case it is too long for the running head
\titlerunning{Lecture Notes in Computer Science: Authors' Instructions}

% the name(s) of the author(s) follow(s) next
%
% NB: Chinese authors should write their first names(s) in front of
% their surnames. This ensures that the names appear correctly in
% the running heads and the author index.
%
\author{Alfred Hofmann%
\thanks{Please note that the LNCS Editorial assumes that all authors have used
the western naming convention, with given names preceding surnames. This determines
the structure of the names in the running heads and the author index.}%
\and Ursula Barth\and Ingrid Haas\and Frank Holzwarth\and\\
Anna Kramer\and Leonie Kunz\and Christine Rei\ss\and\\
Nicole Sator\and Erika Siebert-Cole\and Peter Stra\ss er}
%
\authorrunning{Lecture Notes in Computer Science: Authors' Instructions}
% (feature abused for this document to repeat the title also on left hand pages)

% the affiliations are given next; don't give your e-mail address
% unless you accept that it will be published
\institute{Springer-Verlag, Computer Science Editorial,\\
Tiergartenstr. 17, 69121 Heidelberg, Germany\\
\mailsa\\
\mailsb\\
\mailsc\\
\url{http://www.springer.com/lncs}}

%
% NB: a more complex sample for affiliations and the mapping to the
% corresponding authors can be found in the file "llncs.dem"
% (search for the string "\mainmatter" where a contribution starts).
% "llncs.dem" accompanies the document class "llncs.cls".
%

\toctitle{Lecture Notes in Computer Science}
\tocauthor{Authors' Instructions}
\maketitle


\begin{abstract}

%Mention why convergence in PSO is a problem and why position
%diversity is essential to fix that problem.  

\end{abstract}


\section{Introduction}
\label{introduction}

Mention the hypothesis of the increase of position diversity and the random parameter.

\section{Related Work}
\label{related-work}

\section{Preliminaries}
\label{preliminaries}

EvoSpace

\section{Proposed Method}
\label{proposed-method}

The proposed method is designed to accelerate the convergence of the
PSO algorithm. The hypothesis that was taken into consideration for this method is
that extending PSO to a parallel architecture should increase the
position diversity of the particles, and avoid premature convergence around a local
optimum and should yield better results because of this, as explained
by S. Cheng in \cite{cheng2013population}. Furthermore, each device in the parallel
architecture that is searching for a solution is configured with a
random set of parameters, as proposed by Y. Gong and
A. Fukunaga in \cite{gong2011distributed}. The intention of this approach is to
replace an optimization process of the parameters for a single instance of
an evolutionary algorithm, and, instead, use a number of instances with
different random parameters.  %Don't understand this - JJ
As has been demonstrated in practice, as
in \cite{garcia2014randomized}, \cite{gong2011distributed}, and
\cite{tanabe2013evaluation}, this approach obtains similar results to
an evolutionary algorithm with optimised parameters. % This probably
                                % should go to state of the art - JJ

The PSO algorithm was adapted to integrate it with the EvoSpace
architecture \cite{garcia2015evospace}. EvoSpace enables different
instances of PSO to be working on a single population, taking random
samples of particles.

\begin{figure}
  \centering
  \includegraphics[height=6.2cm]{pdf/distributed-pso}
  \caption{}
  \label{distributed-pso}
\end{figure}

\begin{figure}
  \centering
  \includegraphics[height=6.2cm]{pdf/traditional-pso}
  \caption{}
  \label{traditional-pso}
\end{figure}

\section{Experiments}
\label{experiments}

\section{Results}
\label{results}

\section{Conclusions}
\label{conclusions}

\section{Future Work}
\label{future-work}

\section{Acknowledgements}

We acknowledge support from 
Spanish Ministry of Economy and Competitiveness and European Regional
Development Fund (FEDER) under project EphemeCH
(TIN2014-56494-C4-3-P and  
from University of Granada, PROY-PP2015-06 (Plan Propio 2015 UGR)
\section{Cites}


\cite{garcia2015evospace}
\cite{garcia2014randomized}
\cite{merelo2012pool}
\cite{mussi2011gpu}
\cite{merelo2013designing}
\cite{merelo2008asynchronous}
\cite{morrison2001measurement}
\cite{roy2009distributed}
\cite{sherry2012flex}
\cite{tanabe2013evaluation}
\cite{de1990analysis}
\cite{melin2013optimal}
\cite{zadeh1988fuzzy}
\cite{zadeh1965fuzzy}
\cite{abdelbar2005fuzzy}
\cite{kenndy1995particle}
\cite{mcnabb2007parallel}
\cite{venter2006parallel}
\cite{koh2006parallel}
\cite{cheng2013population}
\cite{gong2011distributed}

\bibliographystyle{splncs03}
\bibliography{paper}

\end{document}
